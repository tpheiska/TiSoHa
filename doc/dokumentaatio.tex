\documentclass[12pt,twoside,a4paper,leqno,titlepage]{article}

\makeatletter
\newcommand{\ps@pagenumberfoot}{
\renewcommand{\@oddfoot}{\hfil\thepage}
   \renewcommand{\@evenfoot}{\thepage\hfil}
}
\makeatother

\input{Tyyli.sty}
\pagestyle{pagenumberfoot}
\title{Dokumentaatio}
\author{Tomi Heiskanen}

\begin{document}

\maketitle

\ 
\thispagestyle{empty}
\newpage

\setcounter{page}{1}
\tableofcontents
\thispagestyle{empty}

\newpage
\section{Johdanto}

Työn aiheena on kurssikysely. Kurssikyselyn avulla kerätään oppilaiden
mielipiteitä ja ideoita kurssin kehittämiseksi. Näiden avulla opettaja voi
kehittää omaa opetusta ja käytettäviä menetelmiä. Oppilaat voivat vastata
kyselyyn nimettömästi.

Työ toteutetaan Helsingin yliopiston tietojenkäsittelytieteen laitoksen users
palvelimella Apache-palvelimen alla. Web-sovelluksen alustajärjestelmän tulee
tukea php-kieltä ja PostgreSQL-tietokantaa.

\section{Käyttötapaukset}

Oppilas haluaa antaa palautetta kurssista ja opetuksesta ja hän voi mennä www-
sivuille ja kertoa sen nimiettömästi. Ohjelma tarkistaa oppilaan statuksen, eli
onko hän kirjoilla opiskelijana ja onko hän kurssilla opiskelijana. Kurssin ulko-
puolinen ei voi antaa palautetta kurssista. Kurssin pitäjä saa lukea palautteet,
mutta hän ei voi yhdistää palautetta kehenkään oppilaaseen. Ulkopuoliset eivät
voi lukea palautteita, mutta on mahdollista saada tietoa palautteiden määrästä
ja voidaan laskea suhde palautteiden määrä/kurssille osallistjat.

\end{document}